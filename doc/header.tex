\usepackage[utf8x]{inputenc}
\usepackage[T2A]{fontenc}
\usepackage[english, russian]{babel}

% Опционно, требует  apt-get install scalable-cyrfonts.*
% и удаления одной строчки в cyrtimes.sty
% Сточку не удалять!
\usepackage{cyrtimes}

% Картнки и tikz
\usepackage{graphicx}
\usepackage{tikz}
\usetikzlibrary{snakes,arrows,shapes}


% Некоторая русификация.
\usepackage{misccorr}
\usepackage{indentfirst}
\renewcommand{\labelitemi}{\normalfont\bfseries{--}}

% Увы, поля придётся уменьшить из-за листингов.
\topmargin -1cm
\oddsidemargin -0.5cm
\evensidemargin -0.5cm
\textwidth 17cm
\textheight 24cm

\sloppy

% Оглавление в PDF
\usepackage[
bookmarks=true,
colorlinks=true, linkcolor=black, anchorcolor=black, citecolor=black, menucolor=black,filecolor=black, urlcolor=black,
unicode=true
]{hyperref}

\usepackage{listings}

% Значения по умолчанию
\lstset{
  basicstyle=\ttfamily,
  columns=fullflexible,
  breaklines=true,       % переносить длинные строки
%  inputencoding=koi8-r,
  showspaces=false,      % показывать пробелы подчеркиваниями -- идиотизм 70-х годов
  showstringspaces=false,
  showtabs=false,        % и табы тоже
  stepnumber=1,
  tabsize=4,              % кому нужны табы по 8 символов?
  frame=single
}

% Свой язык для задания грамматик в BNF
\lstdefinelanguage[]{Pixel}[]{}{
  morekeywords={TRANSACTION_ID,ADVERTISER_ID,OFFER_CODE,OFFER_CODE1,OFFER_CODE2,OFFER_CODE3,COST},
}[keywords,comments,strings]

% Для исходного кода в тексте
\newcommand{\Code}[1]{\texttt{#1}}

\newcommand{\heymoose}{<<HeyMoose!>>}